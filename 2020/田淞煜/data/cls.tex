\begin{conclusion}

恶意的逆向分析技术严重威胁着软件的版权安全,本文针对Windows平台下x86架构下的可执行文件进行了软件保护技术方面的探索,提出了一种基于虚拟机加壳的多样化Handler动态保护方法和一种建立索引的方式对函数间基本块进行交换的静态保护算法。本文从代码层面上对算法进行了阐述,描述了软件保护机制的实现方法,给出了虚拟机加壳技术和静态二进制混淆技术的核心汇编源码,并对设计的核心数据结构进行了说明,本论文的工作总结如下:

首先,PE 文件加壳技术的研究。首先对 PE 文件的加载过程和 PE 文件的文 件结构进行分析,在虚拟机加壳的基础上,提出了多样 化 Handler 虚拟机加壳技术,对同一个 Handler 进行多样化的指令实现,使得同一个功能模块有不同的二进制指令实现,有效了增加逆向分析者的分析成本。 

其次,PE 二进制混淆技术的研究。首先对二进制混淆技术进行了分析,总结 常见二进制混淆方法的不足,提出了建立索引进行函数间基本块交换的混淆算 法,提出了对 PE 文件进行基本块交换后的重构 方法。

最后,根据提出的 虚拟机加壳技术和二进制混淆算法设计并实现了一个PE 文件保护系统,系统包括 PE 文件虚拟机加密壳和 PE 文件二进制混淆器。对该系 统的设计与实现进行了详细阐述,并对 PE 文件虚拟机加密壳和 PE 文件二进 制混淆器进行了功能的验证和性能评估。实验结果表明提出的多样化 Handler 虚 拟机加壳方式和二进制代码混淆算法均有效,有效增强了 Windows 的 PE 文 件的抗逆向分析能力。


  
\end{conclusion}


% \begin{table}[H]
%   \bicaption{中文}{english}
%   \aboverulesep=0ex
%   \belowrulesep=0ex
%   \begin{tabularx}{\columnwidth}{@{} C|C @{}}
%       \toprule[1.5pt]
%       Key                 & Value    \\
%       \hline
%       simulation duration & 12 hours \\
%       \hline
%       update interval     & 1s       \\
%       \hline
%       time-to-live        & 12 hours \\
%       \hline
%       buffer size         & infinite \\
%       \hline
%       message interval    & 20s      \\
%       \bottomrule[1.5pt]
%   \end{tabularx}
%   \label{table: simulation parameters}
% \end{table}