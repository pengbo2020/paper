\chapter{传感器的标定}
\label{cha:calibrations}

\subsection{相机的标定}
\label{cam_calib}
视觉传感器因感光元件有异而分为CMOS相机和CCD相机,由于CMOS受其成像原理所限,无法像CCD相机那样同时接收并处理所有光信号,即卷帘快门产生的“果冻”现象\cite{solutions2011shutter}。但全局快门CMOS相机通过为每个像素点增加了采样保持单元,使其可以同时接收并处理所有光信号,消除了“果冻”效应,可以更好地保留瞬态成像的原本面貌。而相机由于在生产的过程中,不同批次和工艺都会产生一定的差异,故在实际使用之前需要对其进行标定。标定是通过一系列标准的办法估计出相机的一些关键指标参数、外部参数和畸变参数。关键指标参数即所谓的相机内部参数,包括相机的焦距、像元大小和光学中心坐标;外部参数是因为相机的自身转动和平移产生的变化;而畸变参数是由于透镜的存在产生的。

在相机的标定上,卷帘快门和全局快门的相机二者要区分对待。由于卷帘快门因其在移动成像过程中会发生畸变,对其标定需要根据传感器的电子快门时间进行精确的建模,而后对畸变进行矫正。

卷帘快门的标定实际上是对其线延迟进行计算,并尽可能地使计算结果逼近它的真实值。如Luc Oth等人的方法\cite{oth2013rolling},是从一般的投影方程出发,提出了一个适用于卷帘快门相机的连续时间透视投影模型,并给出了相关的透视定位理论。在推导了线延迟估计方程之后,将其与透视姿态方程相结合,基于一组已知的标志点同时估计摄像机姿态和线延迟。最后,通过推导重投影误差的协方差矩阵,建立了标准的最大似然估计量。由于实际SLAM应用中较少考虑使用卷帘快门相机,故不在此详述方法。

而对于全局快门相机,则无需考虑上述内容,直接进行相机的参数标定。

一般,我们使用针孔模型为单个相机的数学模型,如\ref{cha2:equ:cam0}所示。本节所述的所有内容均建立在相机的参考坐标系中。

\begin{equation}
  \label{cha2:equ:cam0}
  s\begin{bmatrix}
    u\\
    v\\
    1
  \end{bmatrix}
  = \mathbf{A}[\mathbf{R}|\mathbf{t}]\mathbf{M}'
\end{equation}

其中

\begin{equation}
  \mathbf{A} = \begin{bmatrix}
    f_x & 0   & c_x\\
    0   & f_y & c_y\\
    0   & 0   & 1
  \end{bmatrix},\ 
  [\mathbf{R}|\mathbf{t}] = \begin{bmatrix}
    r11 & r12 & r13 & t1\\
    r21 & r22 & r23 & t2\\
    r31 & r32 & r33 & t3
  \end{bmatrix},\ 
  \mathbf{M}' = \begin{bmatrix}
    X\\
    Y\\
    Z\\
    1
  \end{bmatrix}
\end{equation}

$\mathbf{A}$矩阵是相机的内部参数矩阵,其中$f$代表焦距,$c$代表光心;$[\mathbf{R}|\mathbf{t}]$是相机在世界坐标系下的旋转和平移矩阵,也被称为相机的外部参数矩阵;$\mathbf{M}'$是图像中的点在世界参考系下对应的三维坐标。由于成像的过程是将三维坐标中的点投影在二维平面上,这个过程丢失了深度维度,也对投影的像进行了尺度化转换,即将真实坐标点转化为像的二维坐标点。

\begin{equation}
  \label{cha2:eqa:xy2uv}
  \begin{aligned}
    &u = f_x * x' + c_x\\
    &v = f_y * y' + c_y
  \end{aligned}  
\end{equation}

其中

\begin{equation}
  \begin{cases}
    x' = \frac {x} {z}\\
    y' = \frac {y} {z}
  \end{cases},\ \begin{bmatrix}
    x\\
    y\\
    z
  \end{bmatrix} = \mathbf{R}\begin{bmatrix}
    X\\
    Y\\
    Z
  \end{bmatrix} + \mathbf{t}
\end{equation}

但实际情况下,由于直接使用针孔相机会导致进光量的急剧减少,故常常我们使用透镜代替针孔的位置,但这一方法引入了畸变,也常被称为失真。透镜畸变从效果上分为两种,分别是桶形畸变和枕形畸变,如图所示。在桶形失真中,图像放大倍率随其像素位置与光轴的距离而减小。明显的效果是在球体(或桶)周围映射的图像的效果。鱼眼镜头采用半球形视图,利用这种畸变将无限宽的物平面映射到有限的图像区域。在变焦镜头镜筒中,畸变出现在镜头焦距范围的中间,在该范围的广角端最严重。枕形畸变则相反,在枕形畸变中,图像放大率随其像素位置与光轴的距离而增加。可见的效果是,未穿过图像中心的线像枕形一样向内弯曲,朝向图像中心。综合两种畸变的效果还会产生一种畸变,我们称之为胡子形畸变。其名字由于其特殊的畸变形态而来,在偏向中心的位置,其畸变于桶形畸变相似,而向外的位置却又与枕形畸变相似。

但是,无论是哪种畸变,一般都可以利用公式\ref{equ:chap2:distortion01}和公式\ref{equ:chap2:distortion02}共同表示。

\begin{equation}
\label{equ:chap2:distortion01}
\begin{aligned}
x'' = x_d &+ (x_d - x_c)(k_1r^2 + k_2r^4 \dots)\\&+(p_1(r^2 + 2(x_d - x_c)^2)\\&+ 2p_2(x_d - x_c)(y_d - y_c))(1 + p_3r^2 + p_4r^4 \dots)
\end{aligned}
\end{equation}
\begin{equation}
\label{equ:chap2:distortion02}
\begin{aligned}
y'' = x_d &+ (x_d - x_c)(k_1r^2 + K_2r^4 + \dots)\\&+(2p_1(x_d - x_c)(y_d - y_c) \\&+ p_2(r^2 + 2(y_d - y_c)^2))(1 + p_3r^2 + p_4r^4 \dots)
\end{aligned}
\end{equation}

在公式\ref{equ:chap2:distortion01}和公式\ref{equ:chap2:distortion02}中,$x_d$和$y_d$分别是镜头在x和y方向的像平面投影的畸变像点,$x_u$和$y_u$是理想针孔相机投射出的未畸变的像点,$x_c$和$y_c$是光心,$k_n$和$p_n$分别是在径向和切向的$n$阶畸变系数,而$r$是$\sqrt{(x_d - x_c)^2 + (y_d - y_c)^2}$。将x$x''$和$y''$分别带入公式\ref{cha2:eqa:xy2uv}中并替换$x'$和$y'$,便可将畸变参数加入模型中。

\begin{equation}
\label{cha2:equ:cam1}
  \begin{aligned}
    &u = f_x * x'' + c_x\\
    &v = f_y * x'' + c_y
  \end{aligned}
\end{equation}

此外,由于某些使用场景的原因,相机的图像传感器可能是由于需要聚焦于斜面而倾斜,这会导致$x$和$y$的透视变形,这种变形可以通过\ref{cha2:equ:cam2}的模型来表示。

\begin{equation}
\label{cha2:equ:cam2}
  s\begin{bmatrix}
    x'''\\
    y'''\\
    1
  \end{bmatrix} = \begin{bmatrix}
    \mathbf{R}_{33}(\tau_x,\tau_y) & 0 & -\mathbf{R}_{13}(\tau_x,\tau_y)\\
    0 & \mathbf{R}_{33}(\tau_x,\tau_y) & -\mathbf{R}_{23}(\tau_x,\tau_y)\\
    0 & 0 & 1
  \end{bmatrix} \mathbf{R}(\tau_x,\tau_y)\begin{bmatrix}
    x''\\
    y''\\
    1
  \end{bmatrix}
\end{equation}

其中,$\tau_x$和$\tau_y$分别是两个方向的角度参数,而$\mathbf{R}(\tau_x,\tau_y)$如\ref{cha2:equ:cam3}所示。

\begin{equation}
\label{cha2:equ:cam3}
\begin{aligned}
  \mathbf{R}(\tau_x,\tau_y) &= \begin{bmatrix}
    \cos(\tau_y) & 0 & -\sin(\tau_y)\\
    0 & 1 & 1\\
    \sin(\tau_y) & 0 & \cos(\tau_y)
  \end{bmatrix}\begin{bmatrix}
    1 & 0 & 0\\
    0 & \cos(\tau_x) & \sin(\tau_x)\\
    0 & -\sin(\tau_x) & \cos(\tau_x)
  \end{bmatrix}\\ &= \begin{bmatrix}
    \cos(\tau_y) & \sin(\tau_y)\sin(\tau_x) & -\sin(\tau_y)\cos(\tau_x)\\
    0 & \cos(\tau_x) & \sin(\tau_x)\\
    \sin(\tau_y) & -\cos(\tau_y)\sin(\tau_x) & \cos(\tau_y)\cos(\tau_x)
  \end{bmatrix} 
\end{aligned}
\end{equation}

最后,需将\ref{cha2:equ:cam1}中的$x''$和$y''$替换为\ref{cha2:equ:cam2}中的$x'''$和$y'''$,如\ref{cha2:equ:cam4}所示。

\begin{equation}
\label{cha2:equ:cam4}
\begin{aligned}
  &u = f_x * x''' + c_x\\
  &v = f_y * y''' + c_y
\end{aligned}
\end{equation}

至此,所有需要通过标定获得的相机参数已定,如\ref{cha2:equ:campara}所示,$k$值的数量一般根据相机透镜的畸变程度而定,如畸变较大的相机可取值到6个,而普通相机的取值数量一般是2到4个。

\begin{equation}
\label{cha2:equ:campara}
  (f_x,f_y,c_x,c_y,k_1,k_2,p_1,p_2,k3,k4,k5,k6,s1,s2,s3,s4,\tau_x,\tau_y)
\end{equation}

一般我们使用标定板辅助相机的参数标定,为了保证标定效果,需要尽可能地让相机成像的所有位置都拍摄过标定板。标定板按照图案可分为棋盘板、方块板、正六边形板和圆点板,需要注意的是,标定板的不同会产生不同的标定结果,但相互之间误差不会大到影响对相机的使用和计算。大多数正常使用的情况,如果不对相机本体造成破坏性的打击或剧烈地冲击,这些相机的参数不会随之发生改变。

经过上述标定后,相机发布的数据才可以被正式使用,数据格式为YUV/MJPEG,帧率和分辨率等依实际传感器及其ISP性能为准。

\subsection{IMU的标定}
\label{imu_calib}
不论是在SLAM技术,还是其他机器人学应用的过程中,IMU都是十分关键的传感器。它提供了自身的位姿变化,并可以通过积分计算获取位姿变化轨迹,轨迹和位姿精度与传感器的选型和计算的滤波算法等都有关。一般在机器人上,我们不考虑使用昂贵的光纤陀螺仪或者庞大的机械陀螺仪获取角速度,而是使用微小灵敏的微机电式的陀螺仪、加速度计和电子罗盘。但这类传感器的问题在于误差较大,需要经过较好的标定和滤波才可以使用。IMU的传感器的误差主要有轴偏差、尺度因子、和偏置与噪声\cite{tedaldi2014robust}。其中加速度计和陀螺仪的测量模型可表示为:

\begin{equation}
  \begin{aligned}
    &\mathbf{a}^{B} = \mathbf{T}^{a}\mathbf{K}^{a}(\mathbf{a}^{S}+\mathbf{b}^{a}+\mathbf{\nu}^{a})\\
    &\mathbf{\omega}^{B} = \mathbf{T}^{g}\mathbf{K}^{g}(\mathbf{\omega}^{S}+\mathbf{b}^{g}+\mathbf{\nu}^{g})
  \end{aligned}
\end{equation}

其中,所有的上标$a$代表与加速度计相关的量,所有的上标$g$代表与陀螺仪相关的量;上标$B$表示正交的参考系,上标$S$是非正交的选准系;$T$是表示轴偏差的矩阵,$K$表示的尺度因子,$\mathbf{b} \in \mathbb{R}^3$和$\nu$分别是偏置和白噪声。

对于加速度计的尺度因子可表示为:

\begin{equation}
  \mathbf{K}^{a} = \begin{bmatrix}
    s^{a}_{x} & 0 & 0\\
    0 & s^{a}_{y} & 0\\
    0 & 0 & s^{a}_{z}
  \end{bmatrix}
\end{equation}

对于陀螺仪的尺度因子可表示为:

\begin{equation}
  \mathbf{K}^{g} = \begin{bmatrix}
    s^{g}_{x} & 0 & 0\\
    0 & s^{g}_{y} & 0\\
    0 & 0 & s^{g}_{z}
  \end{bmatrix}
\end{equation}

对于加速度计的轴偏矩阵可表示为:

\begin{equation}
\label{cha2:equ:imu_tbias_a}
\mathbf{T}^{a} = \begin{bmatrix}
    1 & -\alpha_{yz} & \alpha_{zy}\\
    0 & 1 & -\alpha_(zx)\\
    0 & 0 & 1
  \end{bmatrix}
\end{equation}

对于陀螺仪的轴偏矩阵可表示为:

\begin{equation}
  \label{cha2:equ:imu_tbias_g}
  \mathbf{T}^{g} = \begin{bmatrix}
    1 & -\gamma_{yz} & \gamma_{zy}\\
    \gamma_{xz} & 1 & -\gamma_{zx}\\
    -\gamma_{xy} & \gamma_{yx} & 1
  \end{bmatrix}
\end{equation}

式\ref{cha2:equ:imu_tbias_a}和\ref{cha2:equ:imu_tbias_g}中的$\alpha_{ij}$和${\gamma_{ij}}$分别表示从$i$轴到$j$轴加速度计和陀螺仪的旋转量。

建立传感器的误差模型完毕后,我们可以对其进行依次标定。首先应通过对不同标准姿态的IMU进行静止状态的加速度计值和陀螺仪值的数据采集,标准状态指的是IMU的自坐标系X-Y-Z轴任意一轴与世界坐标系的Y轴平行。一般采集会持续几十次,即旋转几十次IMU传感器并重复采集各个姿态时传感器传回的数据,一般次数越多,计算的效果越好。静止通过式\ref{cha2:equ:static}判断。

\begin{equation}
\label{cha2:equ:static}
  \sigma(t_{w}) = \sqrt{[var_{t_{w}}(\mathbf{a}^{t}_{x})]^{2}+var_{t_{w}}(\mathbf{a}^{t}_{y})]^{2}+var_{t_{w}}(\mathbf{a}^{t}_{z})]^{2}}
\end{equation}

其中,$var_{t_{w}}$代表加速度$\mathbf{a}^{t}$在时间段$t_w$内的方差。判断时,只需通过计算$\sigma(t_{w}) - \sigma(T_{init})$即可,大于零则动态,小于等于则为静态。一般$T_{init}$取50。我们忽略白噪声,为加速度计和陀螺仪建立代价函数,分别有式\ref{cha2:equ:costa}和\ref{cha2:equ:costg}。

\begin{equation}
\label{cha2:equ:costa}
\mathbf{L}(\mathbf{\theta} ^{acc}) = \sum^{M}_{k=1}(||\mathbf{g}||^{2}-||h(\mathbf{a}^{S}_{k},\mathbf{\theta}^{acc})||^{2})^{2}
\end{equation}

其中$\mathbf{\theta} ^{acc}$是加速度计的待求解参数,为:

\begin{equation}
  \mathbf{\theta}^{acc}=[\alpha_{yz},\alpha_{zy},\alpha_{zx},s^{a}_{x},s^{a}_{y},s^{a}_{z},b^{a}_{x},b^{a}_{y},b^{a}_{z}]
\end{equation}

对于陀螺仪的代价函数,有:

\begin{equation}
\label{cha2:equ:costg}
\mathbf{L}(\mathbf{\theta} ^{gyro})=\sum^{M}_{k=2}||\mathbf{u}_{a,k}-\mathbf{u}_{g,k}||^{2}
\end{equation}

其中,$\mathbf{\theta} ^{gyro}$是陀螺仪的待求解参数,$\mathbf{u}_{a,k}$是加速度计在旋转后实际测量的重力值,$\mathbf{u}_{g,k}$如\ref{cha2:equ:integ}是通过积分陀螺仪数据之后得到的新的重力向量,它们分别为:

\begin{equation}
  \mathbf{\theta} ^{gyro}=[\gamma_{yz},\gamma_{zy},\gamma_{xz},\gamma_{zx},\gamma_{xy},\gamma_{yx},s^{g}_{x},s^{g}_{y},s^{g}_{z}]
\end{equation}

\begin{equation}
\label{cha2:equ:integ}
\mathbf{u}_{g,k}=\mathbf{\Psi}[w^{S}_{i},\mathbf{u}_{a,k-1}]
\end{equation}

最后,通过Levenberg–Marquardt算法求出参数即可。

而对于电子罗盘的数据标定,则需要在数据融合时进行实时修正。因这种传感器对环境的磁场非常敏感,当处于没有产生较强磁场的物体的环境中,我们通常可以更信赖它的数据,但当周围放置了能够产生较强磁场的物体后,如电机、音响、耳机和麦克风等,电子罗盘的数据会随之产生漂移,这与传感器自身的误差无关,因为环境中的磁场确实受到了干扰发生了改变,而电子罗盘所测得数据也是真实的。所以在工作前校准的结果,是不能保证运行时能够正常工作的。故在建立节点时,我们不对电子罗盘进行校准。

经过上述校准后,IMU发布的数据才可以被使用,数据发布的顺序分别是X-Y-Z轴的加速度,X-Y-Z轴的角速度和X-Y-Z方向的磁力值,更新频率依实际硬件为准。